% !TEX encoding = UTF-8
% !TEX program = xelatex

\documentclass[a4paper,11pt]{article}

%%% packages--------------------------------------------------------------------------------------
% Language setting
% Replace `english' with e.g. `spanish' to change the document language
\usepackage[english]{babel}

% Set page size and margins
% Replace `letterpaper' with`a4paper' for UK/EU standard size
%\usepackage[letterpaper,top=2cm,bottom=2cm,left=3cm,right=3cm,marginparwidth=1.75cm]{geometry}

% Useful packages
\usepackage{amsmath} 
\usepackage{amssymb}
\usepackage{amstext}
\usepackage{amsfonts}
\usepackage{graphicx}
\usepackage[colorlinks=true, allcolors=blue]{hyperref}
%\usepackage{booktabs}

\usepackage{float}

\usepackage{graphicx} %Loading the package
%\graphicspath{{../../_figures/}}
% \graphicspath{{../../_tikz_plot/}}


\usepackage{lecturenoteStyle_2023}



% \usepackage[sorting=none, backend=bibtex]{biblatex}
% \addbibresource{bib.bib}
 
 

%%%%%%%%%%%%%%%%%%%%%%%%%%%%%%%%%%%%%%%%%%%%%%%%%%
\title{\bf Report Template}
\author{Provide your names in alphabetic order}
\date{}




%%%%%%%%%%%%%%%%%%%%%%%%%%%%%%%%%%%%%%%%%%%%%%%%%%
\begin{document}



\maketitle


%\vspace{-2cm}



%%%%%%%%%%%%%%%%%%%%%%%%%%%%%%%%%%%%%%%%%%%%%
\section{Introduction}

In the introduction, the following contents (but not limited to) should be included
\begin{itemize}
\item Your dictionary learning model, please provide references related to your model;
\item The optimization algorithm, please provide references relate to your algorithms;
\item Basic summary of your obtained results.
\end{itemize}

\begin{remark}
The above contents are basically ``Task 1''.
\end{remark}


\begin{figure}[!ht]
\centering
\subfloat[Barbara]{ \includegraphics[width=55mm]{blank.png} }  
\subfloat[Cameraman]{ \includegraphics[width=55mm]{blank.png} }  
\subfloat[Lena]{ \includegraphics[width=55mm]{blank.png} }  \\ 
%%%%%%%%%%%%%%%
\caption{Dictionary learned from three grayscale images.}
\label{fig:dl_gray}
\end{figure}



\section{Numerical results}

For each task, we provide below the template of summarizing your result and how we rank the results. 

%\vskip3mm

%\begin{figure}[!ht]
%\centering
%\includegraphics[width=58mm]{blank.png}  \\ 
%%%%%%%%%%%%%%%%
%\caption{Caption texts.}
%\label{fig:mcm}
%\end{figure}
%
%
%\begin{figure}[!ht]
%\centering
%\subfloat[Barbara]{ \includegraphics[width=42mm]{blank.png} }  
%\subfloat[Cameraman]{ \includegraphics[width=42mm]{blank.png} }  
%\subfloat[Lena]{ \includegraphics[width=42mm]{blank.png} }  \\ 
%%%%%%%%%%%%%%%%
%\caption{Caption texts.}
%\label{fig:grayimg}
%\end{figure}



\subsection{Task 3: color images denoising}

{\color{red} {\bf ALERT:} the noised images for denoising are now updated, in {\tt mat} file format, please use the new data in the {\tt project.zip} file.} 


For color image denoising, please summarize your result in Table \ref{tab:result_task3}. The result in the table is what I obtained using the sample code provided in the {\tt project.zip} file (of course extended to the color images). 

\begin{remark}
The parameters I used are not tuned for each image, so in principle the results you obtain should be BETTER than mine.
\end{remark}

\begin{table}[h]
\centering
\caption{PSNR values of 18 McM images.}
\label{tab:result_task3}
\begin{tabular}{|c|c|c|c|c|}
\hline
 & Red Channel & Green Channel & Blue Channel & Average of three \\ \hline
McM01 & 27.96 & 27.42 & 27.02 & 27.02 \\ \hline
McM02 & 31.75 & 32.68 & 31.34 & 31.34 \\ \hline
McM03 & 31.47 & 31.16 & 30.18 & 30.18 \\ \hline
McM04 & 34.06 & 34.79 & 32.94 & 32.94 \\ \hline
McM05 & 33.27 & 32.64 & 31.34 & 31.34 \\ \hline
McM06 & 33.80 & 33.13 & 32.82 & 32.82 \\ \hline
McM07 & 32.66 & 32.80 & 32.02 & 32.02 \\ \hline
McM08 & 34.68 & 35.08 & 34.69 & 34.69 \\ \hline
McM09 & 32.47 & 33.23 & 32.87 & 32.87 \\ \hline
McM10 & 32.68 & 33.02 & 32.72 & 32.72 \\ \hline
McM11 & 31.99 & 31.29 & 32.65 & 32.65 \\ \hline
McM12 & 34.53 & 33.79 & 33.63 & 33.63 \\ \hline
McM13 & 35.06 & 35.75 & 33.82 & 33.82 \\ \hline
McM14 & 33.66 & 34.32 & 32.51 & 32.51 \\ \hline
McM15 & 32.08 & 33.49 & 33.13 & 33.13 \\ \hline
McM16 & 30.30 & 28.47 & 30.69 & 30.69 \\ \hline
McM17 & 30.22 & 30.28 & 30.52 & 30.52 \\ \hline
McM18 & 31.45 & 31.02 & 33.35 & 33.35 \\ \hline
\end{tabular}
%
\end{table}



In comparison, the psnr values of the {\bf NEW} noisy images are provided in Table \ref{tab:noisy_task3}. 

\begin{table}[h]
\centering
\caption{PSNR values of 18 noised McM images.}
\label{tab:noisy_task3}
\begin{tabular}{|c|c|c|c|c|}
\hline
 & Red Channel & Green Channel & Blue Channel & Average of three \\ \hline
McM01 & 22.10 & 22.08 & 22.12 & 22.12 \\ \hline
McM02 & 22.09 & 22.12 & 22.12 & 22.12 \\ \hline
McM03 & 22.11 & 22.12 & 22.11 & 22.11 \\ \hline
McM04 & 22.10 & 22.12 & 22.14 & 22.14 \\ \hline
McM05 & 22.11 & 22.10 & 22.12 & 22.12 \\ \hline
McM06 & 22.10 & 22.11 & 22.12 & 22.12 \\ \hline
McM07 & 22.10 & 22.12 & 22.13 & 22.13 \\ \hline
McM08 & 22.10 & 22.12 & 22.11 & 22.11 \\ \hline
McM09 & 22.11 & 22.11 & 22.09 & 22.09 \\ \hline
McM10 & 22.10 & 22.13 & 22.12 & 22.12 \\ \hline
McM11 & 22.12 & 22.11 & 22.11 & 22.11 \\ \hline
McM12 & 22.13 & 22.11 & 22.10 & 22.10 \\ \hline
McM13 & 22.10 & 22.09 & 22.12 & 22.12 \\ \hline
McM14 & 22.11 & 22.10 & 22.10 & 22.10 \\ \hline
McM15 & 22.12 & 22.10 & 22.13 & 22.13 \\ \hline
McM16 & 22.09 & 22.09 & 22.11 & 22.11 \\ \hline
McM17 & 22.11 & 22.11 & 22.10 & 22.10 \\ \hline
McM18 & 22.12 & 22.12 & 22.13 & 22.13 \\ \hline
\end{tabular}
%
\end{table}

\paragraph{Ranking} 
How we rank the result: For each image, we will rank the {\bf average PSNR} value result of each image, the final rank of a group is based on the weighted sum of its ranking of the 18 images. 


\clearpage






\subsection{Task 4: unknown ground truth}


{\color{red} {\bf ALERT:} the noised images for denoising are now updated, in {\tt mat} file format, please use the new data.} 


For color image denoising, please summarize your result in the following table. The result in the table is what I obtained using the sample code provided in the {\tt project.zip} file. 

\begin{table}[h]
\centering
\caption{PSNR values of 18 McM images.}
\label{tab:result_task4}
\begin{tabular}{|c|c|c|c|c|}
\hline
 & Red Channel & Green Channel & Blue Channel & Average of three \\ \hline
McM01 &  &  &  &  \\ \hline
McM02 &  &  &  &  \\ \hline
McM03 &  &  &  &  \\ \hline
McM04 &  &  &  &  \\ \hline
McM05 &  &  &  &  \\ \hline
McM06 &  &  &  &  \\ \hline
McM07 &  &  &  &  \\ \hline
McM08 &  &  &  &  \\ \hline
McM09 &  &  &  &  \\ \hline
McM10 &  &  &  &  \\ \hline
McM11 &  &  &  &  \\ \hline
McM12 &  &  &  &  \\ \hline
McM13 &  &  &  &  \\ \hline
McM14 &  &  &  &  \\ \hline
McM15 &  &  &  &  \\ \hline
McM16 &  &  &  &  \\ \hline
McM17 &  &  &  &  \\ \hline
McM18 &  &  &  &  \\ \hline
\end{tabular}
%
\end{table}



\paragraph{Ranking} 
How we rank the result: For each image, we will rank the {\bf average PSNR} value result of each image, the final rank of a group is based on the weighted sum of its ranking of the 18 images. 




\clearpage


\section{Conclusion}


%%%%%%%%%%%%%%%%%%%%%%%%%%%%%%%%%%%%%%%%%%%%%%%%
\bibliographystyle{plain}
\bibliography{bib}

\end{document}



